\documentclass[a4paper]{article}
\usepackage[utf8]{inputenc}
\usepackage[brazilian]{babel}

\title{Just Keep Moving: Especificação}
\author{Giordano Rodrigues \and João Pedro Holanda}
\date{}

\begin{document}

\maketitle

\section{Introdução}

Esse documento descreve o jogo \emph{Just Keep Moving}, um \emph{side-scroller} de plataforma 2D, de autoria própria, submetido como projeto da primeira unidade do componente curricular IMD0811 - MOTORES DE JOGOS DIGITAIS da UFRN ministrada pelo professor Isaac Franco Fernandes.

Neste documento, delinearemos os seguintes pontos em cada seção abaixo, respectivamente:
\begin{itemize}
	\item Temática e aparência;
	\item Mecânicas e funcionalidade;
	\item Implementação.
\end{itemize}

\section{Temática e Aparência}

O jogo é um \emph{side-scroller} de plataforma 2D, com elementos de fantasia e terror. Inspirado em jogos retrô, foram escolhidos tons de cor mais escuras, e uma baixa resolução de pixels. Essa estética, junto com o efeito vinheta sobre a tela, é evocativa de um tom mais sombrio, que alinha com a temática do jogo.

O personagem principal consiste em um mago -- ainda sem nome -- cuja arma principal é sua própria magia.

Os controles, também inspirados em jogos antigos, são deliberadamente simples, de forma a simular o uso de joysticks dos consoles antigos.

Os níveis são projetados para estimular suspense ou medo, e os inimigos portam aparências ameaçadores e/ou perturbadoras. Todos os elementos visuais do jogo foram feitos pelos desenvolvedores.

Outras fontes de inspiração para esse jogo incluem \emph{The Binding of Isaac}, jogo \emph{top-down} de temática similar, porém mais macabra, e outros \emph{side-scrollers} mais clássicos como \emph{Super Mario Bros.}

\section{Mecânicas e Funcionalidades}

O jogador controla um personagem que pode se movimentar para os lados (com A e D), pular (com W) e atirar (com espaço). O objetivo é chegar ao final de todos os níveis.

Entre cada nível, o jogador poderá salvar o seu progresso. Os níveis também possuem \emph{checkpoints}, marcadores para os quais o personagem retorna após morrer. Esses marcadores são ativados a medida que o jogador avança no nível.

Ao avançar, aparecerão inimigos que causam dano contínuo ao personagem. Após causar dano suficiente, o personagem morre. O jogador pode se livrar dos inimigos ao atirar neles com a arma. Mas deve-se ter cautela, pois cada inimigo morto torna mais difícil de eliminar o próximo. 

Também existirão coletáveis na forma de cápsulas, que são contabilizados como pontos para o jogador. Tais pontos podem ser utilizados para potencializar os tiros, porém essa ação é mais custosa a cada uso subsequente. Quando o personagem volta da morte, a pontuação é resetada.


\section{Implementação}

O jogo está em processo de implementação no motor de jogos Unity 3D, utilizando o modo 2D. Apesar de ser um projeto sem grandes ambições, os componentes implementados foram projetados com reutilização em mente.
Segue uma lista dos componentes, acompanhados por uma breve descrição de cada:
\begin{itemize}
	\item \texttt{Battler}, permite que uma entidade sofra dano;
	\item \texttt{EnemyAI}, ativa comportamento de inimigo em uma entidade;
	\item \texttt{Firepoint}, permite que uma entidade dispare um objeto com um componente \texttt{Shootable};
	\item \texttt{PlayerAnimation}, controla as animações do personagem principal;
	\item \texttt{PlayerJump}, controla a mecância de pulo do personagem principal;
	\item \texttt{PlayerMovement}, controla o movimento do personagem principal;
	\item \texttt{Shootable}, descreve um objeto que pode ser disparado por uma entidade com um \texttt{Firepoint}.
\end{itemize}



\end{document}